%%%%%%%%%%%%%%%%%%%%%%%%%%%%%%%%%%%%%%%%%%%%%%%%%%%%%%%%%%%%%%%%%%%%%%
% LaTeX Example: Project Report
%
% Source: http://www.howtotex.com
%
% Feel free to distribute this example, but please keep the referral
% to howtotex.com
% Date: March 2011 
% 
%%%%%%%%%%%%%%%%%%%%%%%%%%%%%%%%%%%%%%%%%%%%%%%%%%%%%%%%%%%%%%%%%%%%%%
% How to use writeLaTeX: 
%
% You edit the source code here on the left, and the preview on the
% right shows you the result within a few seconds.
%
% Bookmark this page and share the URL with your co-authors. They can
% edit at the same time!
%
% You can upload figures, bibliographies, custom classes and
% styles using the files menu.
%
% If you're new to LaTeX, the wikibook is a great place to start:
% http://en.wikibooks.org/wiki/LaTeX
%
%%%%%%%%%%%%%%%%%%%%%%%%%%%%%%%%%%%%%%%%%%%%%%%%%%%%%%%%%%%%%%%%%%%%%%
% Edit the title below to update the display in My Documents
\title{Project Report}
%
%%% Preamble
\documentclass[paper=a4, fontsize=11pt]{scrartcl}
\usepackage[T1]{fontenc}
\usepackage{fourier}
\usepackage[english]{babel}                                                         % English language/hyphenation
\usepackage[protrusion=true,expansion=true]{microtype}  
\usepackage{amsmath,amsfonts,amsthm} % Math packages
\usepackage[pdftex]{graphicx}   
\usepackage{url}
%%% Custom sectioning
\usepackage{sectsty}
\allsectionsfont{\centering \normalfont\scshape}
%%% Custom headers/footers (fancyhdr package)
\usepackage{fancyhdr}
\pagestyle{fancyplain}
\fancyhead{}                                            % No page header
\fancyfoot[L]{}                                         % Empty 
\fancyfoot[C]{}                                         % Empty
\fancyfoot[R]{\thepage}                                 % Pagenumbering
\renewcommand{\headrulewidth}{0pt}          % Remove header underlines
\renewcommand{\footrulewidth}{0pt}              % Remove footer underlines
\setlength{\headheight}{13.6pt}
%%% Equation and float numbering
\numberwithin{equation}{section}        % Equationnumbering: section.eq#
\numberwithin{figure}{section}          % Figurenumbering: section.fig#
\numberwithin{table}{section}               % Tablenumbering: section.tab#
%%% Maketitle metadata
\newcommand{\horrule}[1]{\rule{\linewidth}{#1}}     % Horizontal rule
\title{
        %\vspace{-1in}  
        \usefont{OT1}{bch}{b}{n}
        \normalfont \normalsize \textsc{Hochschule Bonn Rhein Sieg} \\ 
        \horrule{0.5pt} \\[0.4cm]
        \huge AI Assignment 5\\
        \horrule{2pt} \\[0.5cm]
}
\author{
        \normalfont                                 \normalsize
        Sushma Devaramani\\
        Mihir Patil\\  
        November 19, 2017
}
\date{}
%%% Begin document
\begin{document}
\begin{itemize}
\item Breadth-first search is a special case of uniform-cost search: When all step costs are equal, g(n) is just a multiple of depth n. Thus,breadth-first search and uniform-cost search would behave the same in this case.\\
\item Breadth-first search, depth-first search, and uniform-cost search are special cases of Greedy Best-First Search:$$BFS: f(n) = depth(n)$$\\$$DFS: f(n) = -depth(n)$$\\$$UCS: f(n) = g(n)$$
\item Uniform-cost search is a special case of A* search:\\
$$A* search: f(n) = g(n) + h(n)$$\\
$$Uniform-cost search: f(n) = g(n)$$\\
$$Thus, for h(n) = 0, uniform cost search will produce the same result as A* search.$$
\end{itemize}

\begin{itemize}
\item When is A* complete?\\
A* is complete if it retrns a solution in cases where a solution exists and doesn't return a solution when none exist.Also it must work on all possible inputs.
\item When does A* end the search process?\\
It ends the search process when it finds a goal with the least cost.
\end{itemize}

\begin{itemize}
	\item \textbf{The heuristic manhatten distance is consistent.}\\
	\item \textbf{The heuristic misplaced tiles is admissible but not consistent.}
\end{itemize}
%%% End document
\end{document}