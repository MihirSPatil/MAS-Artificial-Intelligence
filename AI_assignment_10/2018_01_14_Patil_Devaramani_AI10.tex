\documentclass[11pt, a4paper]{report}
\usepackage{adjustbox}
\usepackage[margin=0.5in]{geometry}
\title{Artificial Intelligence for Robotics \\Assignment 10}
\author{Mihir Patil\\Sushma Devaramani}

\begin{document}

\begin{titlepage}
\maketitle
\end{titlepage}

\begin{enumerate}

\item\begin{itemize}
Describe the following concepts in the context of logical reasoning as precisely and compact as possible.
\item \textbf{Enumeration:} One of the methods used for Propositional inference is \textit{Enumeration}. For a knowledge base \textit{KB} it checks if a sentence \textit{$\alpha$} is entailed, by enumerating over all possible models for the KB. It checks if the sentence $\alpha$ is true in all models where KB is true.

\item \textbf{Validity:} A sentence \textit{$\alpha$} is said to be valid if it is true in all models otherwise it is invalid. Valid sentences are necessarily \textit{True}.

\item \textbf{Satisfiability:} A sentence is satisfiable if it is true(or satisfied by) in some model. Satisfiability can be checked by enumerating over the possible models until one if found to be true.

\item \textbf{CNF and DNF:} \begin{itemize}
\item A sentence expressed as a conjunction of clauses is said to be in \textit{Conjunctive Normal Form(CNF)}.\\ e.g: (A $\vee$ $\neg$ B) $\bigwedge$ ($\neg$C $\vee$ D)

\item A sentence expressed as a disjunction of conjunctions of one or more literals is said to be in \textit{Disjunctive Normal Form(DNF)}. e.g: (A $\bigwedge$ $\neg$ B) $\vee$ ($\neg$C $\bigwedge$ D)

\end{itemize} 

\item \textbf{Resolution:} It is a single \textit{Inference rule}, which when combined with a \textit{Complete search algorithm } will yield a \textit{Complete inference algorithm}.

\end{itemize}

\item 
\begin{flushleft}

[(Food $\Rightarrow$ Party) $\vee$ (Drinks $\Rightarrow$ Party)]$ \Rightarrow$ [(Food $\vee$ Drinks) $ \Rightarrow$ Party]
\end{flushleft} 
\begin{itemize}
\item[a.] The above given implication is valid and satisfiable, this can be shown by \textit{Enumeration}. 


\begin{table}[h]
\centering
\begin{adjustbox}{width=\textwidth}

\begin{tabular}{|l|l|l|l|l|l|l|l|l|}
\hline
Food(Fd) & Party(P) & Drinks(D) & Fd $\Rightarrow$ P & D $\Rightarrow$ P & (Fd $\Rightarrow$ P)$\vee$(D $\Rightarrow$ P) & Fd $\wedge$ D & (Fd $\wedge$ D) $\Rightarrow$ P & [(Fd $\Rightarrow$ P)$\vee$(D $\Rightarrow$P)] $\Rightarrow$ [(Fd $\wedge$ D) $\Rightarrow$ P] \\ \hline
F        & F        & F         & T                  & T                 & T                                          & F             & T                             & T                                                                                        \\ \hline
F        & F        & T         & T                  & F                 & T                                          & F             & T                             & T                                                                                        \\ \hline
F        & T        & F         & T                  & T                 & T                                          & F             & T                             & T                                                                                        \\ \hline
F        & T        & T         & T                  & T                 & T                                          & F             & T                             & T                                                                                        \\ \hline
T        & F        & F         & F                  & T                 & T                                          & F             & T                             & T                                                                                        \\ \hline
T        & F        & T         & F                  & F                 & F                                          & T             & F                             & T                                                                                        \\ \hline
T        & T        & F         & T                  & T                 & T                                          & F             & T                             & T                                                                                        \\ \hline
T        & T        & T         & T                  & T                 & T                                          & T             & T                             & T                                                                                        \\ \hline
\end{tabular}
\end{adjustbox}
\caption{Truth Table}
\end{table}

\item[b.] \begin{itemize}


\item \textbf{Right Hand Side:}\begin{center}
(Food $\Rightarrow$ Party) $\vee$ (Drinks $\Rightarrow$ Party)\\
($\neg$ Food $\vee$ Party)$\vee$($\neg$Drinks$\vee$Party)\\
$\neg$ Food $\vee$ Party $\vee$ $\neg$Drinks$\vee$Party\\
$\neg$ Food $\vee$ $\neg$Drinks $\vee$ Party
\end{center}

\item \textbf{Left Hand Side:}\begin{center}
(Food $\wedge$ Drinks) $ \Rightarrow$ Party)\\
$\neg$(Food $\wedge$ Drinks) $\vee$ Party\\
$\neg$ Food $\vee$ $\neg$Drinks $\vee$ Party
\end{center}
\item \textit{ Since both the left and right hand side are the same(equal) they have the same truth value.}
\end{itemize}

\item[c.] Prove your answer to (a) using resolution\\

Propositional resolution is a rule of inference to propositional logic. It is used to build a algorithm proof that is \textit{sound} and \textit{complete} for all the proposition logic.\\

\textbf{Steps for Proposition resolution algorithm:}\\

1. Convert all the sentences to CNF(Conjunctive Normal Form).\\
2. Negate the desired conclusion (the one converted to CNF).\\
3. Apply the resolution rule until,
\begin{enumerate}
\item Derive a empty clause/false(contradiction).
\item Can't apply the rule any more.
\end{enumerate}


\newpage
Some conversion rules used:

1. a $\Rightarrow$ b   $\longrightarrow$   $\neg$ a $\vee$ b \\
2. a $\Longleftrightarrow$ b $\longrightarrow$ ($\neg$ a $\vee$ b) $\wedge$ ( a $\vee$ $\neg$ b)\\
3. $\neg \neg$ a $\longrightarrow$ a\\
4. $\neg$( a $\wedge$ b ) $\longrightarrow$ ($\neg$ a $\vee$ $\neg$ b). (De-Morgan's Law)\\
5. $\neg$( a $\vee$ b ) $\longrightarrow$ ($\neg$ a $\wedge$ $\neg$ b). (De-Morgan's Law)\\

The given implication is,

[(Food $\Rightarrow$ Party) $\vee$ (Drinks $\Rightarrow$ Party)]$ \Rightarrow$ [(Food $\vee$ Drinks) $ \Rightarrow$ Party]

This can be written as,
\begin{equation}
	[(F \Rightarrow P) \vee (D \Rightarrow P)]\Rightarrow [(F \vee D)	\Rightarrow P] 
\end{equation}
\begin{equation}
\neg[(\neg F \vee P) \vee (\neg D \vee P)] \vee (\neg ( F \wedge D) \vee P]
\end{equation}
\begin{equation}
\neg(\neg F \vee P) \wedge \neg ( \neg D \vee P) \vee ( \neg F \vee \neg D \vee P)
\end{equation}
\begin{equation}
(F \wedge \neg P \wedge D) \vee ( \neg F \vee \neg D \vee P)
\end{equation}
Negate the sentence,
\begin{equation}
\neg [(F \wedge \neg P \wedge D) \vee ( \neg F \vee \neg D \vee P)]
\end{equation}
\begin{equation}
\neg [(F \wedge \neg P \wedge D)] \wedge \neg( \neg F \vee \neg D \vee P)]
\end{equation}
\begin{equation}
(\neg F \vee P \vee \neg D) \wedge ( F \wedge D \wedge \neg P)
\end{equation}

The clauses are,

\begin{equation}
\lbrace \neg F , P, \neg D \rbrace 
\end{equation}

\begin{equation}
\lbrace F \rbrace 
\end{equation}
\begin{equation}
\lbrace D \rbrace 
\end{equation}
\begin{equation}
\lbrace \neg P \rbrace 
\end{equation}
Eq (11), (8) leads to 
\begin{equation}
\lbrace \neg F , \neg D \rbrace 
\end{equation}
Eq (12), (9) leads to,
\begin{equation}
\lbrace \neg D \rbrace
\end{equation}
From eq (13) and (10), it leads to empty clause
\begin{equation}
\lbrace \rbrace
\end{equation}

This proves that the sentence is satisfiable and valid.
\end{itemize}

\item Let A, B and C be propositional formulas such that A and B entails C (e.g. $ A \wedge B \longrightarrow C)$. Then, is $ (A \longrightarrow C) \vee (B \longrightarrow C)$ always true?

\textit{solution:}
In order to prove if the sentence entails the knowledge base ($KB\longrightarrow \alpha$), we have to prove that $KB \wedge \neg \alpha$ is \underline{unsatisfiable}, that is the result should return empty clause.

The knowledge base/premise is converted to CNF,

$$A \wedge B \longrightarrow C$$
$$\neg(A\wedge B) \vee C$$
$$ \neg A \vee \neg B \vee C $$
The clauses are,
$\lbrace \neg A \rbrace, \lbrace \neg B \rbrace, \lbrace C \rbrace ---- (a)$ 

Now, convert the conclusion/sentence into CNF and negate it,
$$ (A \longrightarrow C)\vee (B \longrightarrow C)$$
$$\neg[(\neg A \vee C) \vee (\neg B \vee C)]$$
$$\neg (\neg A \vee C) \wedge \neg(\neg B \vee C)$$
$$(A \wedge \neg C) \wedge (B \wedge \neg C)$$
The clauses are,
$\lbrace A \rbrace, \lbrace B \rbrace, \lbrace \neg C \rbrace ---- (b)$

Applying the inference rule on (a) and (b) will result in an empty clause. Hence it can be proved that the sentence $(A\rightarrow C) \vee (B \rightarrow C)$ logically follows the knowledge base $(A \wedge B) \rightarrow C)$ and is always true.

\item Prove the following formulas
\begin{itemize}

\item[a.] $\neg P \wedge \neg Q \Longleftrightarrow \neg ( P \vee Q)$

\begin{equation}
[\neg P \wedge \neg Q \Rightarrow \neg ( P \vee Q)] \wedge [\neg(P \vee Q) \Rightarrow \neg P \wedge \neg Q] 
\end{equation}
\begin{equation}
[P \vee Q \vee \neg(P \vee Q)] \wedge [ \neg (P \vee Q) \vee \neg P \wedge \neg Q]
\end{equation}
\begin{equation}
[P \vee Q \vee \neg (P \vee Q)] \wedge [P \vee Q \vee \neg P \wedge \neg Q]
\end{equation}
\begin{equation}
[P \vee Q \vee \neg P \wedge \neg Q)] \wedge [P \vee Q \vee \neg P \wedge \neg Q]
\end{equation}
After converting to CNF,
\begin{equation}
P \vee Q \vee (\neg P \wedge \neg Q)
\end{equation}
Negate the above sentence,
\begin{equation}
\neg P \wedge \neg Q \wedge (P \vee Q)
\end{equation}
The clauses are,
\begin{equation}
\lbrace P, Q \rbrace
\end{equation}
\begin{equation}
\lbrace \neg P \rbrace
\end{equation}
\begin{equation}
\lbrace \neg Q \rbrace
\end{equation}
From eq (21) and (22),
\begin{equation}
\lbrace Q \rbrace
\end{equation}
From eq (24) and (23), it results to a empty clause,
\begin{equation}
\lbrace  \rbrace
\end{equation}
This proves the formula.\\

\item[b.] $\neg (P \wedge Q) \Longleftrightarrow \neg P \vee \neg Q$

\begin{equation}
[\neg( P \wedge Q) \Rightarrow \neg P \vee \neg Q)] \wedge [\neg P \vee \neg Q \Rightarrow \neg (P \wedge Q)] 
\end{equation}
\begin{equation}
\neg [(\neg P \wedge Q) \vee \neg P \vee \neg Q)] \wedge [ \neg (\neg P \vee \neg Q) \vee \neg (P \wedge Q]
\end{equation}
\begin{equation}
[P \wedge Q \vee \neg P \vee \neg Q)] \wedge [(P \wedge Q) \vee \neg P \vee \neg Q]
\end{equation}

After converting to CNF,
\begin{equation}
(P \wedge Q)\vee \neg P \vee \neg Q
\end{equation}
Negate the above sentence,
\begin{equation}
(\neg P \vee \neg Q) \wedge P \wedge Q)
\end{equation}
The clauses are,
\begin{equation}
\lbrace \neg P, \neg Q \rbrace
\end{equation}
\begin{equation}
\lbrace  P \rbrace
\end{equation}
\begin{equation}
\lbrace  Q \rbrace
\end{equation}
From eq (31) and (32),
\begin{equation}
\lbrace Q \rbrace
\end{equation}
From eq (34) and (33), it results to a empty clause,
\begin{equation}
\lbrace  \rbrace
\end{equation}
This proves the formula.\\

\item[c.] $ P \vee (P\wedge Q) \Longleftrightarrow P$

Converting the above sentence to CNF,
\begin{equation}
P \vee (P \wedge Q)\Rightarrow P \wedge P \Rightarrow P \vee (P \wedge Q)
\end{equation}
\begin{equation}
\{\neg [P \vee (P\wedge Q)] \vee P\} \wedge \{\neg P \vee [P \vee (P\wedge Q)]\}
\end{equation}

\begin{equation}
\{\neg P \wedge( \neg P\vee \neg Q) \vee P\} \wedge \{\neg P \vee (P \vee P \wedge (P \vee Q)]\}
\end{equation}

\begin{equation}
\{(\neg P \wedge \neg P)\vee(\neg P \wedge \neg Q) \vee P\} \wedge \{(\neg P \vee P) \wedge (P \vee Q)]\}
\end{equation}
\begin{equation}
(\neg P \vee \neg P) \wedge(\neg Q \vee P) \wedge (\neg P \vee P) \wedge(P \vee Q)
\end{equation}
\begin{equation}
 \neg P \wedge(\neg Q \vee P) \wedge (\neg P \vee P) \wedge(P \vee Q)
\end{equation}

Negate the above sentence,
\begin{equation}
(P \vee Q) \wedge (\neg P \vee P )\wedge (\neg P \vee \neg P) \wedge \neg Q
\end{equation}
The clauses are,
\begin{equation}
\{  P, Q \}
\end{equation}

\begin{equation}
\{ P, \neg P \}
\end{equation} 

\begin{equation}
\{ \neg Q \}
\end{equation}

\begin{equation}
\{ \neg P \}
\end{equation}

From eq (43) and (45),
\begin{equation}
\{  P \}
\end{equation}
From eq (44), (47) and (46), we get empty clause
\begin{equation}
\{ \}
\end{equation}

This proves the given sentence
\end{itemize}
\end{enumerate}

\end{document}